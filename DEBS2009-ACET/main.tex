\documentclass{sig-alternate}

\begin{document}

\conferenceinfo{DEBS}{2009 Nashville, Tennessee USA}

\title{Partial Actor Continuations}
\subtitle{Efficient and Extensible Request Routing for Event-Driven Architectures}

\numberofauthors{1} 

\author{
\alignauthor
Stefan Plantikow\\
       \affaddr{Zuse Institute Berlin (ZIB)}\\
       \affaddr{Takustrasse 7}\\
       \affaddr{14195 Berlin, Germany}\\
       \email{plantikow@zib.de}
}

\date{1 March 2009}


\maketitle

\begin{abstract}

The request routing logic between the different stages in event-driven architectures is often
distributed over different portions of the source code. This can make it hard to change and
understand the flow of events in the system.

The article presents an approach that allows writing request routing logic as a set of routing
scripts. Requests are executed step-wise according to their script by sending partial continuations
that encapsulate their respective request's current execution state to stages for local processing
and optional forwarding of follow-up continuations. The implementation of a simple domain specific
language for routing scripts for the scala actor library is described and evaluated. The results
show that request routing with partial actor continuations performs equally or better to using a
separate stage for request routing logic for scripts of at least 3 sequential steps. 

\end{abstract}

\category{H.2.4}{Information Systems}{Systems}[Concurrency]         
\category{D.1.3}{Software}{Programming Techniques}[Concurrent Programming]         
\category{D.3.3}{Programming Languages}{Language Constructs and Features}[Concurrency]
\category{D.2.11}{Software Engineering}{Distribution, Maintenance, and Enhancement}[Extensibility]

\keywords{Request Routing, Event-Driven Architecture, Partial Continuation, Actor Model, Scala}


\section{Introduction}             

Using a staged, event-driven architecture is an approach to the design of server software that can 
provide high degrees of concurrency and throughput.  This is achieved by structuring the software as
a set of stages that communicate exclusively via event queues and by optionally performing admission 
control on each queue.  

// TODO Examples of staged architectures are ...
// TODO XtreemFS example

Depite their benefits, event-driven architectures can lead to a distribution of application logic
over different stages. The implementation of each stage usually resides in a different portions of
the source code and therefore the dynamic routing of requests through different stages is not
described by a single source location.

This reduces the understandability of the system and in turn makes it harder to modify the request
routing logic. Additionally, it makes it difficult to add new request types without changing the
source code of existing stages and redeploying the system.

// Problem deScription

// Article overview
                                
                         
\section{Preliminaries}

// Actor Model


\section{Request Routing}

The application logic of event-driven architectures can be split into two categories. First,
processing logic, is the part of application logic that necessarily must be executed at a specific
stage in order to access resources or state that are only available locally. Second, the request
routing logic describes how a given incoming request is handled by executing processing logic at
many stages in some order. 

Often, processing logic is contained implicitely in the event handling of single stages and the
follow-up events created by them. It is this implicit containment that can make systems difficult to
understand and modify.

Additionally, request routing logic may be stateful, i.e. the result of executing processing logic
at some stage may determine how and at which stages request handling needs to be continued. This
places further burdens on the implementation of single stages, as incoming- and outgoing events need
to be amended with the necessary state data, although it might be completely independent from the
intended purpose of the stage.

To give an example, imagine a simple system for launching satellites into space. Incoming
reqeuests are amended with authentication information in the first stage. In the second stage, this
information is then used to authorize the request and eventually launch the rocket. Only after the
satellite has begun to operate, some third stage (i.e. the press office) is informed.  Now, imagine
that the initial request needs to be amended with extra information (name and owner of satellite)
for the press office.  Passing this information down requires modifying the events to and from
the rocket launching stage with fields for the additional payload, although this extra information
is of no importance to launching the rocket.

This intertwining of processing and request routing logic is a case of insufficient separation of
concerns, calling for a different way to describe both types of application logic.  Next, two 
different approaches that address this issue are described.


\subsection{Ping-Pong-Approach}

A straightforward way to deal with this problem is to transfer the execution of the request routing
logic to a separate stage. Requests enter the system as separate events at such a routing stage.
This stage then continously forwards events to some stage, waits for a reply, and upon receipt,
decides how to continue based on this and previous reply-events for the request. In the following,
this will be called the ping-pong-approach.

While this approach allows to write the event flow portion of the application logic in a single
stage, it has two disadvantages: First, it requires the creation of additional stages and the
associated computational overhead. Second, and more importantly, it results in extra messages
between request routing and regular processing stages.


\subsection{Partial Actor Continuations}

// PAC - Approach, requires introducing continuations somewhere
              
// Continuations

// Partial Continuations (?)

                         



\section{Sofleuse: A DSL for Request Routing}
    
// Scala and Actors (briefly)
          
// Plays

// Example

// Describe execution

           
\section{Implementation}
                  
// Scala and Actors (as needed for implementation)

// CPS-Transform 


\section{Evaluation}

// Describe settting

// Present results

// Discuss in detail esp w regard to actor library


\section{Related Work}

// Let's see...

\section{Summary}
                  
// Fazit

// Interesting applications

\section{Acknowledgments}

\bibliographystyle{abbrv}
%\bibliography{sigproc}  

\end{document}
